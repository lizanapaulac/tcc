%================================= Resumo e Abstract ========================================
\chapter*{Resumo}


\begin{quotation}
\noindent Este trabalho de conclus�o de curso visou analisar o processo de aprendizagem da l�ngua portuguesa como segunda l�ngua pela crian�a surda, atrav�s dos jogos digitais. Teve como metodologia o desenvolvimento de jogos digitais em uma sala de aula bil�ngue multisseriada para a an�lise do processo de aquisi��o de novos voc�bulos e amplia��o lexical. Este estudo concluiu que os jogos digitais s�o uma importante ferramenta de motiva��o e aprendizagem para os alunos surdos, contribuindo com o aprendizado lexical.

\vspace*{0.5cm}

\noindent Palavras-chave: Educa��o bil�ngue para surdos,  jogos digitais; portugu�s como L2.

\end{quotation}


\chapter*{Abstract}


\begin{quotation}


\noindent This Undergraduate thesis examined the process of learning Portuguese Language as a second language for deaf children, through digital games. The methodology used was the development of digital games in bilingual multisseriate classroom in order to analyze the acquisition of new vocabulary and lexical expansion. This study concludes that digital games are an important tool of motivation and learning to deaf students, contributing with lexical learning.

\vspace*{0.5cm}

\noindent Key-words: Deaf Bilingual Education; Digital Games; Portuguese (L2)

\end{quotation}

\null

